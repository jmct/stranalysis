%\documentclass[justified, twoside, a4paper, symmetric]{tufte-book}
\documentclass{jfp1}

\title{Known When It's Now or Never}

\author{Jos\'{e} Manuel Calder\'{o}n Trilla}


%%%% TODO Package %%%% 
\usepackage[disable,prependcaption,textsize=small]{todonotes}

%\usepackage{amsmath}
\usepackage{xargs}
\usepackage{url}

\usepackage{tikz}

\usepackage{csquotes}


\def \hasalpha {\(\alpha\)}
\def \hasbeta {\(\beta\)}
\def \hasgamma {\(\gamma\)}
\def \hasmu {\(\mu\)}
\def \hasphi {\(\phi\)}
\def \haspi {\(\pi\)}
\def \hasrho {\(\rho\)}
\def \haslambda {\(\lambda\)}

\def \meet {\<meet\> ($\sqcap$) }
%\def \pmeet {\bm{\&}}
\def \join {\<join\> ($\sqcup$) }

\newcommandx{\tocite}[2][1=]{\todo[linecolor=red,backgroundcolor=red!25,bordercolor=red,#1]{Cite: #2}}
\newcommandx{\todoinline}[1]{\todo[inline]{#1}}
\newcommandx{\todofig}[1]{\todo[inline]{Make figure: #1}}

\newcommand{\pmeet}{\&}

\usepackage[square]{natbib}

%%%% Haskell Style by Prof. Chakravarty %%%%
\usepackage{haskell}

%This is the stuff for semantic equations%
%%%%%%%%%%%%%%%%%%%%%%%%%%%%%%%%%%%%%%%%%%
\robustify\bfseries
\newsavebox{\sembox}
\newlength{\semwidth}
\newlength{\boxwidth}

\newcommand{\Sem}[1]{%
\sbox{\sembox}{\ensuremath{#1}}%
\settowidth{\semwidth}{\usebox{\sembox}}%
\sbox{\sembox}{\ensuremath{\left[\usebox{\sembox}\right]}}%
\settowidth{\boxwidth}{\usebox{\sembox}}%
\addtolength{\boxwidth}{-\semwidth}%
\left[\hspace{-0.3\boxwidth}%
\usebox{\sembox}%
\hspace{-0.3\boxwidth}\right]%
}
%%%%%%%%%%%%%%%%%%%%%%%%%%%%%%%%%%%%%%%%%%

%% Tikz styles
%%%%%%%%%%%%%%%%%%%%%%%%%%%%%%%%%%%%%%%%%%%%%%%%%%%%%%%%%%%%%%%%%

\tikzset{
    hasse/.style={shape=circle, scale=0.35, draw}
}

\tikzset{
    hassef/.style={fill=black, shape=circle, scale=0.35, draw}
}

\tikzset{
    center/.style={shape=circle, scale=0.35}
}

%% Sometimes we want to give a definition; this is how we do that
%%%%%%%%%%%%%%%%%%%%%%%%%%%%%%%%%%%%%%%%%%%%%%%%%%%%%%%%%%%%%%%%%
\newcommand{\defineword}[2]{%
\begin{description}%
    \item{\textbf{#1}} \hfill \\%
        {#2}%
\end{description}%
}

\newcommand{\sigval}[1]{\bfseries #1}

%%%% hyperref needs to be last (apparently)
%\usepackage{hyperref}
%
%\hypersetup{
% colorlinks,
% citecolor=Red,
% linkcolor=Black,
% urlcolor=Blue
%}


% TODO get this to work with haskell style (figure 26 and 27 especially)
%\usepackage{float} % for box around figures
%\floatstyle{boxed}
%\restylefloat{figure}

% Compromise for the lack of above
%\DeclareCaptionFormat{myformat}{#1#2#3\hrulefill}
%\DeclareCaptionFormat{fnoline}{#1#2#3}
%\captionsetup[figure]{format=myformat}

\begin{document}

\chapter{Finding Safe Parallelism}
\label{chap:discovery} 
\input{src/Preamble.tex}

    \section{Original Motivation vs. Our Motivation}
    \input{src/Motivation.tex}

    \section{Overview}
    \label{sec:strictnessOverview}
    \input{src/Overview.tex}

    \section{Two-Point Forward Analysis}
    \label{sec:twoPoint}
    \input{src/TwoPoint.tex}

    \section{Four-Point Forward Analysis}
    \label{sec:fourPoint}
    \input{src/FourPoint.tex}

    \section{Projection-Based Analysis}
    \label{sec:projections}
    \input{src/Projections.tex}

    \section{Summary}
    \label{sec:summ2}
    \input{src/Summary.tex}

% For now we aren't talking about derivin strategies
%
%\chapter{Derivation and Use of Parallel Strategies}
%\label{chap:derivation}
%\input{Deriving-Strats/Preamble.tex}
%
%    \section{Expressing Need, Strategically}
%    \label{sec:expressingNeed}
%    \input{Deriving-Strats/ExpressingNeed.tex}
%
%    \section{Deriving Strategies from Projections}
%    \label{sec:derivation}
%    \input{Deriving-Strats/StrategyDerivation.tex}
%
%
%    \section{Using Derived Strategies}
%    \label{sec:parPlacement}
%    \input{Platform/Oracles.tex}
    

\bibliography{literature}
\bibliographystyle{jfp}

\end{document}
