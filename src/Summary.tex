This chapter explored the most common static analyses used for strictness
analysis.  Because the initial placement of parallelism is crucial to our
technique we explored each possible analysis in depth, highlighting the
drawbacks of the two-point and four-point analyses. While projection-based
analysis is significantly more complex, its ability to determine the strictness
properties of functions based on the \emph{demand} placed on their results
provides too many benefits to ignore.

Many of the previous attempts and using strictness analysis for implicit
parallelism used the four-point analysis. Combined with \emph{evaluation
transformers} (discussed in the next section) the four-point analysis is able
to identify a significant amount of parallelism. Unfortunately, the analysis is
limited to functions on lists, which while ubiquitous, significantly restricts
the potential of identifying parallelism in more complex programs.

The projection-based analysis not only provides more insight into functions
like \<append\>, as discussed in Section \ref{sec:projSem}, but it also allows
us to determine a useful set of strictness properties for functions on
arbitrary types. This greatly expands the applicability of the strictness
analysis for finding potential parallelism.

The major drawback of Hinze's projection-based analysis is that we, as compiler
writers, no longer know in advance the set of demands our analysis will return.
With the four-point analysis we can hard-code the parallel Strategies that
correspond to each of the points in the domain. If we expand our analysis to
include pairs, we can again add the corresponding strategies. With the
projection-based analysis we no longer have that foresight. This is the issue
we address in the next chapter.
