As noted in the last section, two-point domains are quite limited when dealing
with lazy structures. A more formal explanation for this limitation is that the
two-point domain really represents reduction to WHNF or $\bot$, and nothing
else. In the case of flat domains this is sufficient because WHNF is all there
is. For nested data types the reality is much different. 

For functions that work on lists, like \<sum\> or \<append\>, strictness up to
WHNF is not much benefit. Strictness analysis as described in the previous
section would be able to tell us that \<sum\> requires its argument to be
defined, allowing us to evaluate it before entering the function (or in
parallel). But it is only safe up to WHNF. Once the first \<Cons\> is reached
we must stop evaluation of the list or risk introducing non-termination.
Through the lens of implicit parallelism it seems that we are unlikely to
benefit from introducing parallel evaluation when we are limited to
WHNF.\footnote{Indeed most uses of the basic strictness information were for
improving the code generation to avoid building unnecessary suspensions.} This
is clearly a problem.

The solution seems clear: we must extend the abstract domains for non-flat data
types so that we can have more detailed strictness information. For some data
types, extending the technique is straightforward. In F-Lite, pairs can be
defined as follows.

\begin{haskell*}
data Pair \hasalpha \hasbeta = MkPair \hasalpha \hasbeta
\end{haskell*}

Many languages, such as Haskell, Clean, and the ML family provide the following
syntactic sugar for pairs (and other $N$-tuples): \<(\hasalpha,\hasbeta)\>.

A first try at representing pairs of values from the two-point domain could
give us the lattice in Figure \ref{fig:unboxedPairs}.

\captionsetup[figure]{format=fnoline}
\begin{figure}
\centering
\begin{subfigure}{0.4\textwidth}
\begin{tikzpicture}
    \node [center] (cent) {};
    \node [hasse, above = of cent, label=above:{\large\<(\(\top\),\(\top\))\>}]  (top)   {};
    \node [hasse, left = of cent, label=left:{\large\<(\(\top\),\(\bot\))\>}]    (left)  {};
    \node [hasse, right = of cent, label=right:{\large\<(\(\bot\),\(\top\))\>}]  (right) {};
    \node [hasse, below = of cent, label=below:{\large\<(\(\bot\),\(\bot\))\>}] (bot)   {};

    \draw[black] (top) -- (left);
    \draw[black] (top) -- (right);
    \draw[black] (right) -- (bot);
    \draw[black] (left) -- (bot);
\end{tikzpicture}
\caption{Unlifted Pairs}
\label{fig:unboxedPairs}
\end{subfigure}
\hfill
\begin{subfigure}{0.4\textwidth}
\begin{tikzpicture}
    \node [center] (cent) {};
    \node [hasse, above = of cent, label=above:{\large\<(\(\top\),\(\top\))\>}]  (top)   {};
    \node [hasse, left = of cent, label=left:{\large\<(\(\top\),\(\bot\))\>}]    (left)  {};
    \node [hasse, right = of cent, label=right:{\large\<(\(\bot\),\(\top\))\>}]  (right) {};
    \node [hasse, below = of cent, label=315:{\large\<(\(\bot\),\(\bot\))\>}] (bot)   {};
    \node [hasse, below = of bot, label=below:{\large$\bot$}] (Bot)   {};

    \draw[black] (top) -- (left);
    \draw[black] (top) -- (right);
    \draw[black] (right) -- (bot);
    \draw[black] (left) -- (bot);
    \draw[black] (bot) -- (Bot);
\end{tikzpicture}
\caption{Lifted Pairs}
\label{fig:boxedPairs}
\end{subfigure}
\caption{Domain for pairs of flat-domain values}
\hrulefill
\end{figure}
\captionsetup[figure]{format=myformat}

The meaning of this lattice is fairly intuitive. When we possess a pair of
flat-domain values there are four possibilities, we can have

\begin{enumerate}
    \item The pair structure itself (\<MkPair\>), but accessing either value
        results in non-termination
    \item The pair structure itself, but accessing the \<fst\> element 
        results in non-termination
    \item The pair structure itself, but accessing the \<snd\> element
        results in non-termination
    \item The pair structure itself and both values are fully defined
\end{enumerate}

Notice that possibilites 2 and 3 are similar in that there are one defined and
one undefined item in each. Suggesting that one of 2 or 3 is more defined than
the other would make little sense.  For this reason we say that they are
\emph{incomparable}, i.e. they are neither more nor less defined than each
other.

However, the lattice in Figure \ref{fig:unboxedPairs} is only valid for
\emph{unlifted} pairs, where the constructor value, \<(,)\>, is always defined.
The reality for non-strict languages is that \emph{any} value may be undefined,
including the constructors for product types.

This means that we must \emph{lift} the domain, adding a further bottom value
that represents a failure to construct the pair's outermost constructor
(\<MkPair\> in the F-Lite case). The result, shown in Figure
\ref{fig:boxedPairs}, is typical of domains for strictness analysis on finite
types. You can construct an appropriate domain assuming that the structure is
itself defined, then lift the resulting lattice with an additional $\bot$ that
represents a failure to construct the structure.

Because this domain is still finite we are able to incorporate it into the
framework developed by Mycroft without much issue, simply defining the
appropriate \<meet\> and \<join\> on the lattice and the strictness properties
for any primitives that work with pairs. The main issue is that extending the
technique to non-flat domains in the obvious way introduces infinite domains
for recursive types, losing a lot of the power of abstract interpretation.
\tocite{John Hughes work on non-flat domains before wadler and mycroft's thesis
pg 82}

The first practical solution was proposed by Wadler involving a four-point
domain for lists \citep{wadler1987strictness}. Instead of representing the
recursive structure of lists directly, which creates an infinite domain, Wadler
chose a domain that represents four degrees of definedness for lists.

\pagebreak

\begin{figure}[t]
\centering
\begin{tikzpicture}
    \node [hasse, label=above:{\large$\top_{\in}$}]                   (full)   {};
    \node [hasse, below = of full, label=right:{\large$\bot_{\in}$}]  (finite) {};
    \node [hasse, below = of finite, label=right:{\large$\infty$}]    (inf)    {};
    \node [hasse, below = of inf, label=below:{\large$\bot$}]         (bot)    {};

    \draw[black] (full) -- (finite);
    \draw[black] (finite) -- (inf);
    \draw[black] (inf) -- (bot);
\end{tikzpicture}
\caption{Wadler's Four-point Domain}
\label{fig:listDomain}
\end{figure}

The result, as shown in Figure \ref{fig:listDomain}, can be described, from least
to most defined as follows:

\begin{enumerate}
    \item $\bot$ represents all undefined lists
    \item $\infty$ represents all undefined lists, lists  with undefined tails
        and all infinite lists
    \item $\bot_{\in}$ represents all of the above in addition to all finite
        lists with at least one undefined element
    \item $\top_{\in}$ represents fully defined lists along with all of the above
\end{enumerate}


Because we are now concerning ourselves with values from different domains in
our analysis we must now know the types of expressions in our program. This
ensures that we do not accidentally try to \<meet\> or \<join\> values from
different domains. 

\begin{figure}[b]
\centering
\begin{minipage}{.5\textwidth}
\begin{haskell*}
cons^{\#} \(\top\) \(\top_{\in}\) &=& \(\top_{\in}\) \\
cons^{\#} \(\top\) \(\bot_{\in}\) &=& \(\bot_{\in}\) \\
cons^{\#} \(\top\) \(\infty\)     &=& \(\infty\) \\
cons^{\#} \(\top\) \(\bot\)       &=& \(\infty\) \\
\end{haskell*}
\end{minipage}
\quad\quad
\begin{minipage}{.5\textwidth}
\begin{haskell*}
cons^{\#} \(\bot\) \(\top_{\in}\) &=& \(\bot_{\in}\) \\
cons^{\#} \(\bot\) \(\bot_{\in}\) &=& \(\bot_{\in}\) \\
cons^{\#} \(\bot\) \(\infty\)     &=& \(\infty\) \\
cons^{\#} \(\bot\) \(\bot\)       &=& \(\infty\)
\end{haskell*}
\end{minipage}
\caption{Definition of $cons^{\#}$ for a Four-Point Domain}
\label{fig:cons4}
\end{figure}


To incorporate the four-point domain into the abstract interpretation from the
previous section we need a few new primitives. The \<Cons\> constructor is
given the abstract definition shown in Figure \ref{fig:cons4}. \<Nil\>, being a
fully defined list, is always abstracted as $\top_{\in}$.  There are a few
points worth mentioning about the definition of \<cons^{\#}\>.  First, none of
the equations result in $\bot$.  This makes sense with our understanding of
lazy evaluation, if we have the outermost constructor we have a value in WHNF
and therefore it is definitely \emph{not} $\bot$. Additionally, notice that
\<cons\>ing a defined value onto $\bot_{\in}$ also results in $\bot_{\in}$,
this keeps \<cons^{\#}\> monotonic in addition to aligning with our intuitions
(\<cons\>ing a defined element to the beginning of a list with possibly
undefined elements does not suddenly make the list fully defined). 

We must therefore alter the $\mathcal{A}$ rules for nullary constructors and
add a pattern for when \<Cons\> is used. The modified \<Con c\> rule and the
new rule for \<Cons\> are shown in Figure \ref{fig:consAI}.

\begin{figure}[t]
\begin{haskell*}
\mathcal{A}\Sem{Con c} \hasphi && \\
\hsbody{%
    \quad\quad  &\mid c == ``Nil'' &= \(\top_{\in}\) \\
    \quad\quad  &\mid otherwise    &= \(\top\)
    } \\
%
\mathcal{A}\Sem{App (Con ``Cons'') [x, xs]} \hasphi &=&
        cons^{\#} (\mathcal{A}\Sem{x} \hasphi) (\mathcal{A}\Sem{xs} \hasphi)\\
%
\end{haskell*}
\caption{Modification to $\mathcal{A}$ for List Constructors}
\label{fig:consAI}
\end{figure}

In addition to \<Cons\> and \<Nil\>, we need to define new interpretations for
\<Case\> expressions. Pattern matching on a value from the four-point domain
will require a different interpretation than the previous section. The new
domain introduces two problems that must be dealt with:

\begin{enumerate}
    \item Abstracting the alternatives of a \<Case\> expression naively can
        approximate too much, making the results less effective
    \item Choosing appropriate approximations for the bindings introduced with
        the \<Cons\> alternative
\end{enumerate}

We will show the solutions to these obstacles one at a time.

\paragraph{Problem 1:} The first point has to do with pattern matching and
preventing the \<Nil\> alternative from weakening our approximations. For now
we will ignore the question of how to approximate the bindings introduced with
the \<Cons\> alternative.

Take the following template for pattern matching on lists:

\begin{haskell*}
\hscase{\hsinf{a list}}{%
    Nil       &\to \hsinf{\<Nil\> branch} \\
    Cons x xs &\to \hsinf{\<Cons\> branch with possible occurrences of \<x\> and \<xs\>}
    }
\end{haskell*}

If we name the \<Nil\> branch \<a^{\#}\> and treat the \<Cons\> branch as a function
\<f^{\#}\> of \<x\> and \<xs\> we have the following form

\begin{haskell*}
\hscase{\hsinf{a list}}{%
    Nil       &\to a^{\#} \\
    Cons x xs &\to f^{\#} x xs
    }
\end{haskell*}

If we were to naively use the \<Case\> rule from the previous section we
would have\footnote{We use the notation found in \citep{turnerHistory} for
abstracting a variable from an expression: \<[x]e\> denotes abstracting
occurrences of \<x\> out of \<e\>. This results in a function equivalent to
\<(\haslambda x \to e)\>.}

\begin{haskell*}
\mathcal{A}\Sem{Case xs [Nil \to e_{1}, Cons y ys \to e_{2}]} \hasphi &=&
    xs^{\#} \(\sqcap\) (a^{\#} \(\sqcup\) f^{\#} \(\top \ \top_{\in}\))
    \hswhere{%
    &xs^{\#} &= \mathcal{A}\Sem{xs} \hasphi \\
    &a^{\#}  &= \mathcal{A}\Sem{e_{1}} \hasphi \\
    &f^{\#}  &= [ys][y](\mathcal{A}\Sem{e_{2}} \hasphi)
    }
\end{haskell*}

The issue is that the \<meet\>ing of \<a^{\#}\> with \<f^{\#} y ys\> will often
prevent the analysis from providing useful information. Take the function
\<sum\> for example:

\begin{haskell*}
sum xs = \hscase{xs}{%
    Nil       &\to 0 \\
    Cons y ys &\to y + sum ys
    }
\end{haskell*}

In this case, \<a^{\#}\> will always be $\top$, when we abstract the function
to get \<xs^{\#} \(\sqcap\) (\(\top\) \(\sqcup\) f^{\#} y ys)\> the result
would only ever be $\bot$ if \<xs \(\equiv \bot\)\>. Therefore it is only safe
for us to evaluate the list up to WHNF, when \<sum\> clearly needs a
fully-defined list. Moreover, because we may lack information about
the definedness of \<xs\> we must be safe and approximate \<y\> and \<ys\> to
the top of their lattices ($\top$ and $\top_{\in}$, respectively). 

Wadler's key insight was that the use of pattern matching allowed us to retain
information that would otherwise be lost when performing abstract
interpretation \citep{wadler1987strictness}. Whereas in our previous abstract
interpretation from Section \ref{sec:twoPoint} we had to \join all of the
branches in a \<Case\> expression, we can now use the fact that we know \<Nil\>
is always the $\top_{\in}$ value in our domain.  Why is this?  Because \<Nil\>
is a fully defined list with \emph{no bottom elements}!

This means that we only have to consider the value of \<a^{\#}\> when our
\<Case\> expression matches on $\top_{\in}$. This prevents the definedness
of \<a^{\#}\> from preventing more accurate approximations for when the list is
less defined than $\top_{\in}$, solving our first issue.

\paragraph{Problem 2:} The second problem was choosing appropriate approximations
for the bindings introduced with the \<Cons\> alternative. Happily, this turns out
to be quite easy to solve.

In our two-point analysis all bindings introduced by an alternative to a
\<Case\> expression are approximated by $\top$ because we do not `know' how to
approximate non-flat structures. When evaluating a \<Case\> on our four-point
domain we can use the knowledge we have of what lists each point in the domain
corresponds to. We can use the definition of the primitive \<cons^{\#}\> as
a lookup table, switching the right hand and left hand sides of each equation.
This gives us the following:

\begin{haskell}
    \(\top_{\in}\) &\to& cons^{\#} \(\top\) \(\top_{\in}\)              \\  
    \(\bot_{\in}\) &\to& (cons^{\#} \(\bot\) \(\top_{\in}\)) \(\sqcup\)
                         (cons^{\#} \(\top\) \(\bot_{\in}\)) \(\sqcup\)
                         (cons^{\#} \(\bot\) \(\bot_{\in}\)) \(\sqcup\) \\  
    \(\infty\)     &\to& (cons^{\#} \(\top\) \(\infty\))     \(\sqcup\)
                         (cons^{\#} \(\top\) \(\bot\))       \(\sqcup\)
                         (cons^{\#} \(\bot\) \(\infty\))      \(\sqcup\)
                         (cons^{\#} \(\bot\) \(\bot\))
\end{haskell}

The absence of a rule for $\bot$ is due to the fact that we would never match
on an undefined value, resulting in $\bot$ regardless of the values of the
alternatives. We can also remove several of the alternatives in the cases for
$\infty$ and $\bot_{\in}$ due to the necessity for the abstraction of the
alternative branches to be monotonic \citep{wadler1987strictness}.\footnote{For
example, with \(\bot_{\in}\) we will always have \(f^{\#} \bot \bot_{\in}
\sqsubseteq f^{\#} \top \bot_{\in}\), making \(f^{\#} \bot \bot_{\in}\)
unnecessary since \(x \sqcup y \equiv y\) when \(x \sqsubseteq y\). Applying
this reasoning to \(\infty\) leaves us with only \(f^{\#} \top \infty\) to
consider.}

\begin{figure}[t!]
\begin{haskell*}
\mathcal{A}\Sem{Case xs [Nil \to e_{1}, Cons y ys \to e_{2}]} \hasphi && \\
\hsbody{%
    \quad\quad &\mid xs^{\#} == \(\top_{\in}\) &= a^{\#} \(\sqcup\) f^{\#} \(\top\ \top_{\in}\) \\
    \quad\quad &\mid xs^{\#} == \(\bot_{\in}\) &= f^{\#} \(\bot\ \top_{\in}\) \(\sqcup\) f^{\#} \(\top\ \bot_{\in}\) \\
    \quad\quad &\mid xs^{\#} == \(\infty\)     &= f^{\#} \(\top\ \infty\)\\
    \quad\quad &\mid otherwise                 &= \bot
} \hswhere{%
    &xs^{\#} &= \mathcal{A}\Sem{xs} \hasphi \\
    &a^{\#}  &= \mathcal{A}\Sem{e_{1}} \hasphi \\
    &f^{\#}  &= [ys][y](\mathcal{A}\Sem{e_{2}} \hasphi)
    }
\end{haskell*}
\caption{Abstraction of Case Expressions on Lists}
\label{fig:absCase}
\end{figure}


Taking these insights into account leaves us with the rule for
pattern matching on lists seen if Figure \ref{fig:absCase}.

\<meet\> and \<join\> are easily defined for the four-point domain. If we
assign each point in the domain a value according to its position in the lattice,
with $\bot$ being $0$ and $\top_{\in}$ being $3$, we can define \meet as \<min\>
and \join as \<max\>.

With everything in place, we can now determine whether an analysis using this
four-point domain is more suitable for implicit parallelism.

\subsubsection{Length and Sum}

We will use the simple recursive \<length\> function for our first example of
using this analysis (\<sum\> is defined similarly, replacing the \<1\> with
\<y\>).

\begin{haskell*}
length xs &=& \hscase{xs}{%
    Nil &\to 0 \\
    Cons y ys &\to 1 + length ys
    }
\end{haskell*}

Because \<length\> and \<sum\> take only one argument, which is a list, we must
analyse the functions at each of the four points in our domain for lists.
Recursion can be dealt with in the same manner as shown in Section
\ref{sec:twoPoint}. Once a fixed-point is reached we are left with the
following results.

\begin{table}[h!]
\centering
\vspace{10pt}
\begin{tabular}{c || c c}
    $xs^{\#}$ & \<length^{\#} xs^{\#}\> & \<sum^{\#} xs^{\#}\> \\
    \hline
    $\top_{\#}$ & $\top$                & $\top$ \\
    $\bot_{\#}$ & $\top$                & $\bot$ \\
    $\infty$ & $\bot$                   & $\bot$ \\
    $\bot$ & $\bot$                     & $\bot$
\end{tabular}    
\caption{Analysis of \(length^{\#}\) and \(sum^{\#}\) Using 4-point Domains}
\label{tab:lengthSum}
\end{table}

The results in Table \ref{tab:lengthSum} are exactly what we would expect. If
\<length\> or \<sum\> are passed infinite lists then program will result in
non-termination. \<sum\> has the additional constraint that all \emph{elements}
of its input list must also be defined. This analysis would allow us to
evaluate the argument to \<sum\> fully, in parallel, making it a significant
improvement to the simple two-point analysis from Section \ref{sec:twoPoint}.

\subsubsection{Discussion of Four-Point Strictness Analysis}

Because lists are one of the most common structures in functional programming,
this development allowed strictness analysis to be useful in a wide variety of
`real' systems. This also makes strictness analysis' use for parallelism more
realistic. We can now tell the machine to evaluate lists in parallel up to the
degree that it is safe to do so. Some of the more successful attempts at
implicit parallelism were based on using this strictness information, most
notably Burn's work on parallelisation of functional programs for a Spineless
$G$-Machine \citep{burn1987evaluation} and the work of
\citet{hogen1992automatic} on automatically parallelising programs for a
distributed reduction machine.

While this four-point domain made strictness analysis much more flexible it
suffers from a few considerable shortcomings:

\begin{enumerate}
    \item An argument is only considered strict for a function if it is strict
        in \emph{all} possible contexts\footnote{In other words, an argument
                for a function is only strict if it is strict for all possible
                uses of that function. We explore this further in the next
                section.} of that function
    \item For other structures similar domains must be \emph{designed}, i.e.
        there does not seem to be straightforward way to derive a `good' finite
        domain for every recursive type
    \item Even when the compiler writer designs additional abstract domains
        for other recursive types, the calculation of fixed points becomes
        prohibitively expensive with more complex abstract domains
\end{enumerate}

To illustrate the first problem we can study the results of applying this
analysis to the \<append\> function, which can be seen in Table
\ref{tab:append}.

\begin{table}[h!]
\centering
\vspace{10pt}
\begin{tabular}{c c || c c c c}
                               & \multicolumn{1}{c}{} & \multicolumn{4}{c}{$ys^{\#}$}            \\
                               &             & $\top_{\in}$ & $\bot_{\in}$ & $\infty$ & $\bot$   \\
    \cline{2-6}
    \multirow{4}{*}{$xs^{\#}$} & $\top_{\#}$ & $\top_{\in}$ & $\bot_{\in}$ & $\infty$ & $\infty$ \\
                               & $\bot_{\#}$ & $\bot_{\in}$ & $\bot_{\in}$ & $\infty$ & $\infty$ \\
                               & $\infty$    & $\infty$     & $\infty$     & $\infty$ & $\infty$ \\
                               & $\bot$      & $\bot$       & $\bot$       & $\bot$   & $\bot$
\end{tabular}    
\caption{Analysis of \(append^{\#}\ xs^{\#}\ ys^{\#}\) Using 4-point Domains}
\label{tab:append}
\end{table}


We can see that while the first list is always strict up to WHNF, the second
list is not strict. This is unfortunate because we know that \<append\> is
strict in both arguments under certain conditions.

For example, if we pass the result of \<append\> to \<length\> then we know
that \emph{both} argument lists for \<append\> must be finite for the result
of the call the \<length\> to terminate. The inability for this analysis
to express that form of strictness is a major weakness.

The limitations due to this first point were well known at the time of Wadler's
paper on the four-point domain. However, the solutions seemed ad-hoc and were
on shaky theoretical grounds \citep{hughes1986strictness, hughes1987analysing}. The
introduction of the four-point domain was successful, in part, due to it fitting
naturally in the strictness analysis techniques that were already understood.
Fortunately, we have the benefit of time and work on the analysis of strictness that
takes into account the \emph{use} of a function, using \emph{projections}, is
much better understood \citep{hinze1995projection, SergeyDemand}.

The need to design a suitable domain for each recursive type is unfortunate.
Ideally the strictness analysis in a compiler would work on whatever types the
programmer decides to define. Functional languages are often lauded for their
ability to have few primitive types and allow the programmer to define their
own `first class' types.  Having strictness analysis that only functions well
on lists subverts this ideal, creating a leaky abstraction. Programmers will
use lists even when inappropriate because the compiler is so much better at
optimising them than any custom type. While not motivated by the second issue,
projection-based analysis solves it anyway, allowing strictness analysis to be
performed on arbitrary types with very few restrictions.

As for the third shortcoming, projection-based analysis does not make
calculating fixed points free. It does however shift the complexity of the
analysis. Instead of being exponential in the number of arguments, it grows
relative to the size of a function's \emph{return} type. While not a panacea in
this regard it does make projection-based analysis practical.

Overall, strictness analysis using the four-point domain is a significant
improvement over a basic two-point domain, particularly for use in exploiting
implicit parallelism. While having solved several of the downsides of using
a simple two-point domain, the four-point analysis still suffers from
significant problems when taking our use-case into account.
